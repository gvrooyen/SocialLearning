\documentclass[a4paper,10pt]{article}
%\documentclass[a4paper,10pt]{scrartcl}

\usepackage{xltxtra}

\setromanfont[Mapping=tex-text]{Linux Libertine O}
% \setsansfont[Mapping=tex-text]{DejaVu Sans}
% \setmonofont[Mapping=tex-text]{DejaVu Sans Mono}

\defaultfontfeatures{Mapping=tex-text}
\setmainfont[SmallCapsFont = Fontin SmallCaps]{Fontin}

\title{Social Learning Strategies Tournament II Entry: BlueGenes}
\author{G-J van Rooyen}
\date{28 February 2012}

\begin{document}
\maketitle

\section{Overview}
\emph{BlueGenes} is a genetically programmed entry to the Social Learning Strategies Tournament II\@.
The model script is mostly machine generated, based on parametric trait algorithms.
The solution experiments with a new idea in genetic programming, where evolution is not based
on a tree of primitives, but rather on a state graph of parametric algorithms (traits) and transition conditions
that can be suggested by the designer.

The entry itself is largly machine generated, and should ideally not be debugged directly. The \texttt{move()}
function is split into eight parts, corresponding to all possible combinations of \texttt{canChooseModel},
\texttt{canPlayRefine} and \texttt{multipleDemes}. Each algorithm inside such a combination has been individually
evolved to best fit the constraints.

Each algorithm is defined as a state graph, which allows an individual to change its behaviour over its lifetime.
This is achieved by defining each state's completion criteria (\texttt{*\_done()} function) based on only its
observed history, and by selecting a current state based on previous states' exit criteria and exit rounds. Since no
local information can be stored by an individual, this information must be built up by the individual during each
invocation.

Furthermore, some traits allow individuals to diversify across their current population. This is achieved by playing
a random move during certain key moments in an individual's history (usually in terms of its state graph). These
random choices can be detected in subsequent rounds, and be used to individualise the agent's behaviour, depending on
that initial choice.

\end{document}
