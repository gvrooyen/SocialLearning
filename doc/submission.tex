\documentclass[a4paper,10pt]{article}
%\documentclass[a4paper,10pt]{scrartcl}

\usepackage{xltxtra}

\setromanfont[Mapping=tex-text]{Linux Libertine O}
% \setsansfont[Mapping=tex-text]{DejaVu Sans}
% \setmonofont[Mapping=tex-text]{DejaVu Sans Mono}

\defaultfontfeatures{Mapping=tex-text}
\setmainfont[SmallCapsFont = Fontin SmallCaps]{Fontin}

\usepackage{multicol}
\RequirePackage{color,graphicx}
\usepackage[usenames,dvipsnames]{xcolor}
\usepackage[big]{layaureo}

\usepackage{hyperref}
\definecolor{linkcolour}{rgb}{0,0.2,0.6}
\hypersetup{colorlinks,breaklinks, urlcolor=linkcolour, linkcolor=linkcolour}

\title{Social Learning Strategies Tournament II Entry: BlueGenes}
\author{G-J van Rooyen\\
{\footnotesize Dept.\ E\&E Engineering, Stellenbosch University, South Africa}}
\date{28 February 2012}

\begin{document}
\maketitle

\section{Overview}
\emph{BlueGenes} is a genetically programmed entry to the Social Learning Strategies Tournament II\@.
The agent script is mostly machine generated, based on parametric trait algorithms.
The solution experiments with a new idea in genetic programming, where evolution is not based
on a tree of primitives, but rather on a state graph of parametric algorithms (traits) and transition conditions
that can be suggested by the designer.

The entry itself is largely machine generated, and should ideally not be debugged directly. The \texttt{move()}
function is split into eight parts, corresponding to all possible combinations of the \texttt{canChooseModel},
\texttt{canPlayRefine} and \texttt{multipleDemes} options. Each algorithm inside such a combination has been individually
evolved to best fit the constraints.

Each algorithm is defined as a state graph, which allows an individual agent to change its behaviour over its lifetime.
This is achieved by defining each state's completion criteria (\texttt{*\_done()} function) based on only its
observed history, and by selecting a current state based on previous states' exit criteria and exit rounds. Since no
local information can be stored by an individual, this information must be built up by the individual during each
invocation, by using the provided history lists.

Furthermore, some traits allow individuals to diversify across their current population. This is achieved by playing
a random move during certain key moments in an individual's history (usually in terms of its state graph). These
random choices can be detected in subsequent rounds, and be used to individualise the agent's behaviour, depending on
that initial choice.

Although the provided agent script is quite long (around 2500 lines of code), much of the code is traits repeated
across the eight mode combinations. It also includes dormant traits that the solution acquired over its evolution,
but which may not ever be reached in this solution. These have been left untrimmed, since they contribute very
little overhead, and the machine-generated code was rather left intact to avoid the manual introduction of errors.
Despite the long code length, an agent will quickly hone in on the relevant code segment during its round,
and the average time taken per round was found to be on par with the reference solution given in the tournament rules.

\section{Traits}
In searching for model solutions, the genetic program randomly combines self-contained algorithms provided by
the programmer. These algorithms are called \emph{traits}, and each trait represents a state in an evolved agent's
state machine. During this state, the trait's algorithm defines the sequence of moves that are played.

Traits' algorithms may depend on parameters that can be optimised over generations, called \emph{evolvables}.
Although a programmer may provide typical starting values, these parameters are evolved over time to improve
the fitness of the agent.

Since traits represent stages in an indiviual agent's state machine (i.e.\ an agent can move through many states in its
lifetime, each represented by a unique trait), the trait definition should include a \texttt{*\_done()} function
to indicate the conditions where a state can transition to a next state. States may have multiple output
conditions, each representing a different outgoing edge on an agent's state graph, and the \texttt{*\_done()}
function should indicate which edge to to take when the state finishes.

This section will provide a brief description of the various traits used to evolve the submitted solution.

\subsection{Pioneering}
This is an initial state, which means that it can only occur at the entry point of an agent's state graph
(and consequently at the start of an individual agent's life). The agent plays OBSERVE on its first round.
If it observes no other agents playing EXPLOIT, it assumes that it is part of the cohort of pioneers --
the very first group of agents alive in a simulation. It then proceeds to play INNOVATE for a number of
rounds before exiting the state. If it determines that it is not a pioneer (i.e.\ it observed other agents
playing EXPLOIT) it exits this state immediately.

\subsubsection*{Evolvables:}
\begin{itemize}
 \item \texttt{N\_rounds}: The number of rounds a pioneer should play INNOVATE.
\end{itemize}

\subsection{PioneeringBi}
This trait is identical to \emph{Pioneering}, except that it defines two exit conditions: if it determines
that the agent is a pioneer, it playes \texttt{N\_rounds} of INNOVATE, and exits via the first output
edge. Otherwise, it exits via the second output edge.

\subsection{DiscreteDistribution}
The simplest strategy: in this state, the agent randomly plays moves, based on an evolvable distribution.
It is a terminal state, which means that it has no output edges. This trait has several duplicates
(\emph{DiscreteDistributionB}, \emph{DiscreteDistributionC}, etc.) which allows the state to occur multiple
times in an agent's state graph.

\subsubsection*{Evolvables:}
\begin{itemize}
 \item \texttt{Pi, Po, Pe, Pr}: Weights defining the probability that INNOVATE, OBSERVE, EXPLOIT and REFINE
   are played. The weights need not be normalised (i.e.\ they need not sum to unity).
\end{itemize}

\subsection{Specialisation}
This single-round state randomly plays a move based on an evolvable distribution. It then exits on one
of four output edges, corresponding to the move chosen. This allows agents across the population to
diversify (i.e.\ move into diverse subsequent states) based on the random distribution.

This trait has a duplicate
(\emph{SpecialisationB}) which allows it to occur twice
in an agent's state graph.

\subsubsection*{Evolvables:}
\begin{itemize}
 \item \texttt{Pi, Po, Pe, Pr}: Weights defining the probability that INNOVATE, OBSERVE, EXPLOIT and REFINE
   are played. The weights need not be normalised (i.e.\ they need not sum to unity).
\end{itemize}

\subsection{ExploitGreedy}
Exploit the maximum-payoff state until the payoff drops below its initial value (increases in payoff are happily
ignored). Then, exit the state (possibly returning again later). This trait has no evolvable parameters.
It has a duplicate
(\emph{ExploitGreedyB}) which allows it to occur twice
in an agent's state graph.

\subsection{Study}
The perfect counterpart to \emph{ExploitGreedy}, the \emph{Study} state allows an agent to learn new acts
for a few rounds. It plays INNOVATE, OBSERVE and REFINE from an evolvable distribution for a number of
rounds, and then exits.

This trait has a duplicate
(\emph{StudyB}) which allows it to occur twice
in an agent's state graph.

\subsubsection*{Evolvables:}
\begin{itemize}
 \item \texttt{N\_rounds}: The number of rounds to remain in this state.
 \item \texttt{Pi, Po, Pr}: Weights defining the probability that INNOVATE, OBSERVE and REFINE
   are played. The weights need not be normalised (i.e.\ they need not sum to unity).
\end{itemize}

\subsection{InnovationBeat}
This is the most intricate trait. Its premise is that a population of agents could maximise information
exchange if they could synchronise their moves in some or other way. Since there is no absolute reference
of round numbers for the individual agents, and the only way that agents can learn about their peers'
actions is through the OBSERVE action, \emph{InnovationBeat} attempts to synchronise agents' behaviour
by forcing regular rounds where no EXPLOIT acts can be observed. This is achieved by letting agents who have
already synchronised, play INNOVATE at regular intervals. New agents can play OBSERVE for a while, and
try to detect the round at which no (or few) models are observed; this is assumed to be the INNOVATE
synchronisation beat. The agent then falls into this rhythm.

This behaviour is achieved by evolving two sequences of moves. Each sequence starts with the INNOVATE
move; the next move (EXPLOIT / OBSERVE) is played randomly, which effectively splits the population into
two groups. Group A continues to play the one sequence, and Group B the other. The remaining acts in
each sequence are chosen from \{EXPLOIT, OBSERVE, REFINE\} so that the two groups never play OBSERVE
simultaneously. Ideally, one group will learn from the other at some points in the sequence.

When an agent enters the \emph{InnovationBeat} state, it plays OBSERVE for \texttt{N\_Seq} moves, picks
the round when the lowest payoff was observed as the synchronisation round, and then picks a group (A or B)
and starts playing its sequence, synchronised to the rest of the population.

\emph{InnovationBeat} is a terminal state; once an agent has entered it, it remains in this state for the
rest of its lifetime.

\subsubsection*{Evolvables:}
\begin{itemize}
 \item \texttt{N\_Seq}: The length of a sequence (i.e.\ the number of rounds between synchronised
    INNOVATE moves).
 \item \texttt{seq\_A, seq\_B}: The evolved move sequences.
 \item \texttt{Pa}: The probability of an agent picking Group A as its own.
\end{itemize}

\subsection{InnovationBeatSpatial}
This variation on \emph{InnovationBeat} takes the \emph{multipleDemes} scenario into account. If an agent
migrates across demes, its earlier synchronisation is no longer valid. In this case, the state exits (potentially
re-entering itself immediately to resynchronise, or perhaps spending a few rounds in the \emph{Study} state).

\section{Submission}
The tournament submission includes:
\begin{itemize}
 \item This document, which provides an overview of the solution.
 \item \texttt{BlueGenes.py}, the submitted solution.
 \item \texttt{individual\_modes.zip}, for information only. This contains the sub-solutions that were merged into
   \texttt{BlueGenes.py}. The filename of the sub-solution indicates whether it applies to \emph{canChooseModel}
   ('O'), \emph{cumulative} ('R'), \emph{multipleDemes} ('D') or some combination thereof.
 \item The full genetic programming framework, including a reference simulator that implements the tournament
   rules to measure an agent's fitness (defined as average total payoff per round) is available at
   \href{https://github.com/gvrooyen/SocialLearning}{https://github.com/gvrooyen/SocialLearning}. All software
   is copyright \copyright\ 2012, Stellenbosch University, but is freely distributable and usable under the
   Academic Free License 3.0.
\end{itemize}


\section{Declaration}
By submitting this entry, I hereby accept the rules contained in the ``Rules for entry'' document provided by
the organisers on the tournament website.

\end{document}
